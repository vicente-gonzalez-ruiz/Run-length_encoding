%%% Local Variables:
%%% mode: latex
%%% TeX-master: "<none>"
%%% End:

\title{\href{https://en.wikipedia.org/wiki/Run-length_encoding}{Run-length
encoding}{Run-length encoding}}
\author{Vicente González Ruiz}

\maketitle

\section{Idea}

\begin{itemize}
\item RLE (Run Length Encoding) is a technique that removes the data
  redundancy produced by the repetition of symbols. Example:

\begin{verbatim}
aaaaa = 5a
\end{verbatim}

\item Depending on the length of the source alphabet and the maximal
  length of the run, different versions of RLE codecs have been
  proposed.
\end{itemize}

\section{\(N\)-ary run length encoding}}

RLE for \(N\)-ary alphabets (alphabets of size \(N\)), where typically,
\(N=256\).

\subsection{Encoder}

\begin{enumerate}
\def\labelenumi{\arabic{enumi}.}
\tightlist
\item While there are symbols to encode:

  \begin{enumerate}
  \def\labelenumii{\arabic{enumii}.}
  \tightlist
  \item
    Let \(s\) the next symbol.
  \item
    Read the next \(n\) consecutive symbols equal to \(s\).
  \item
    Write the pair \(ns\).
  \end{enumerate}
\end{enumerate}

\subsection{Example}

Runs:

\begin{verbatim}
aaaabbbbbaaaaaabbbbbbbcccccc
\end{verbatim}

are encoded as:

\begin{verbatim}
4a5b6a7b6c
\end{verbatim}

\subsection{Decoder}

\begin{enumerate}
\def\labelenumi{\arabic{enumi}.}
\tightlist
\item While there are \(ns\) pairs to decode:

  \begin{enumerate}
  \def\labelenumii{\arabic{enumii}.}
  \tightlist
  \item Write \(n\)-times the symbol \(s\).
  \end{enumerate}
\end{enumerate}

\subsection{Lab: basic RLE}
\href{https://nbviewer.jupyter.org/github/vicente-gonzalez-ruiz/Run-length_encoding/blob/master/RLE.ipynb}{IPython notebook}

\section{Binary run length encoding}

\begin{itemize}
  \tightlist
\item In binary RLE is not necessary to indicate the next symbol (only the
  length) because when a run ends, only the other (possible) symbol will
  start with the next run.
\end{itemize}

\subsection{Encoder}

\begin{enumerate}
\def\labelenumi{\arabic{enumi}.}
\tightlist
\item
  Let \(s\leftarrow\) \texttt{0}.
\item
  While there are bits to encode:

  \begin{enumerate}
  \def\labelenumii{\arabic{enumii}.}
  \tightlist
  \item
    Read the next \(n\) consecutive bits equal to \(s\).
  \item
    Write \(n\).
  \item
    \(s\leftarrow (s+1)~\text{modulus}~2\).
  \end{enumerate}
\end{enumerate}

    \hypertarget{example}{%
\subsubsection{Example}\label{example}}

Runs:

\begin{verbatim}
0000111110000001111111000000
\end{verbatim}

are encoded as:

\begin{verbatim}
4 5 6 7 6
\end{verbatim}

\subsection{Decoder}

\begin{enumerate}
\def\labelenumi{\arabic{enumi}.}
\tightlist
\item
  Let \(s\leftarrow\) \texttt{0}.
\item
  While there are items \(n\) to decode:

  \begin{enumerate}
  \def\labelenumii{\arabic{enumii}.}
  \tightlist
  \item
    Write \(n\) bits equal to \(s\).
  \item
    \(s\leftarrow (s+1)~\text{modulus}~2\).
  \end{enumerate}
\end{enumerate}

\section{\href{https://en.wikipedia.org/wiki/Microcom_Networking_Protocol\#MNP_5}{MPN-5 run length encoding}}

\begin{itemize}
\tightlist
\item
  Created by
  \href{https://en.wikipedia.org/wiki/Microcom_Networking_Protocol}{Microcom
  Inc.} for the
  \href{https://scholar.google.es/scholar?hl=es\&as_sdt=0\%2C5\&q=held+data+compression+techniques+applications\&btnG=}{MNP
  (Microcom Networking Protocol) 5}.
\end{itemize}

\subsection{Codec}

\begin{verbatim}
Input     Output
--------- ---------
ab        ab
aab       aab
aaab      aaa0b /* 3-symbols length run == <ESC> code-word */
aaaab     aaa1b
aaaaab    aaa2b
:         :
a^nb      aaa(n-3)b
\end{verbatim}

    \hypertarget{example}{%
\subsubsection{Example}\label{example}}

Runs:

\begin{verbatim}
aaaabbbbbaaaaaabbbbbbbccccccb
\end{verbatim}

are encoded as:

\begin{verbatim}
aaa1bbb2aaa3bbb4ccc3b
\end{verbatim}

\subsubsection{Lab: MPN-5}
\href{https://nbviewer.jupyter.org/github/vicente-gonzalez-ruiz/Run-length_encoding/blob/master/MPN-5.ipynb}{IPython notebook}
